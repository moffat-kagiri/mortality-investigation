% Options for packages loaded elsewhere
\PassOptionsToPackage{unicode}{hyperref}
\PassOptionsToPackage{hyphens}{url}
%
\documentclass[
]{article}
\usepackage{amsmath,amssymb}
\usepackage{iftex}
\ifPDFTeX
  \usepackage[T1]{fontenc}
  \usepackage[utf8]{inputenc}
  \usepackage{textcomp} % provide euro and other symbols
\else % if luatex or xetex
  \usepackage{unicode-math} % this also loads fontspec
  \defaultfontfeatures{Scale=MatchLowercase}
  \defaultfontfeatures[\rmfamily]{Ligatures=TeX,Scale=1}
\fi
\usepackage{lmodern}
\ifPDFTeX\else
  % xetex/luatex font selection
\fi
% Use upquote if available, for straight quotes in verbatim environments
\IfFileExists{upquote.sty}{\usepackage{upquote}}{}
\IfFileExists{microtype.sty}{% use microtype if available
  \usepackage[]{microtype}
  \UseMicrotypeSet[protrusion]{basicmath} % disable protrusion for tt fonts
}{}
\makeatletter
\@ifundefined{KOMAClassName}{% if non-KOMA class
  \IfFileExists{parskip.sty}{%
    \usepackage{parskip}
  }{% else
    \setlength{\parindent}{0pt}
    \setlength{\parskip}{6pt plus 2pt minus 1pt}}
}{% if KOMA class
  \KOMAoptions{parskip=half}}
\makeatother
\usepackage{xcolor}
\usepackage[margin=1in]{geometry}
\usepackage{color}
\usepackage{fancyvrb}
\newcommand{\VerbBar}{|}
\newcommand{\VERB}{\Verb[commandchars=\\\{\}]}
\DefineVerbatimEnvironment{Highlighting}{Verbatim}{commandchars=\\\{\}}
% Add ',fontsize=\small' for more characters per line
\usepackage{framed}
\definecolor{shadecolor}{RGB}{248,248,248}
\newenvironment{Shaded}{\begin{snugshade}}{\end{snugshade}}
\newcommand{\AlertTok}[1]{\textcolor[rgb]{0.94,0.16,0.16}{#1}}
\newcommand{\AnnotationTok}[1]{\textcolor[rgb]{0.56,0.35,0.01}{\textbf{\textit{#1}}}}
\newcommand{\AttributeTok}[1]{\textcolor[rgb]{0.13,0.29,0.53}{#1}}
\newcommand{\BaseNTok}[1]{\textcolor[rgb]{0.00,0.00,0.81}{#1}}
\newcommand{\BuiltInTok}[1]{#1}
\newcommand{\CharTok}[1]{\textcolor[rgb]{0.31,0.60,0.02}{#1}}
\newcommand{\CommentTok}[1]{\textcolor[rgb]{0.56,0.35,0.01}{\textit{#1}}}
\newcommand{\CommentVarTok}[1]{\textcolor[rgb]{0.56,0.35,0.01}{\textbf{\textit{#1}}}}
\newcommand{\ConstantTok}[1]{\textcolor[rgb]{0.56,0.35,0.01}{#1}}
\newcommand{\ControlFlowTok}[1]{\textcolor[rgb]{0.13,0.29,0.53}{\textbf{#1}}}
\newcommand{\DataTypeTok}[1]{\textcolor[rgb]{0.13,0.29,0.53}{#1}}
\newcommand{\DecValTok}[1]{\textcolor[rgb]{0.00,0.00,0.81}{#1}}
\newcommand{\DocumentationTok}[1]{\textcolor[rgb]{0.56,0.35,0.01}{\textbf{\textit{#1}}}}
\newcommand{\ErrorTok}[1]{\textcolor[rgb]{0.64,0.00,0.00}{\textbf{#1}}}
\newcommand{\ExtensionTok}[1]{#1}
\newcommand{\FloatTok}[1]{\textcolor[rgb]{0.00,0.00,0.81}{#1}}
\newcommand{\FunctionTok}[1]{\textcolor[rgb]{0.13,0.29,0.53}{\textbf{#1}}}
\newcommand{\ImportTok}[1]{#1}
\newcommand{\InformationTok}[1]{\textcolor[rgb]{0.56,0.35,0.01}{\textbf{\textit{#1}}}}
\newcommand{\KeywordTok}[1]{\textcolor[rgb]{0.13,0.29,0.53}{\textbf{#1}}}
\newcommand{\NormalTok}[1]{#1}
\newcommand{\OperatorTok}[1]{\textcolor[rgb]{0.81,0.36,0.00}{\textbf{#1}}}
\newcommand{\OtherTok}[1]{\textcolor[rgb]{0.56,0.35,0.01}{#1}}
\newcommand{\PreprocessorTok}[1]{\textcolor[rgb]{0.56,0.35,0.01}{\textit{#1}}}
\newcommand{\RegionMarkerTok}[1]{#1}
\newcommand{\SpecialCharTok}[1]{\textcolor[rgb]{0.81,0.36,0.00}{\textbf{#1}}}
\newcommand{\SpecialStringTok}[1]{\textcolor[rgb]{0.31,0.60,0.02}{#1}}
\newcommand{\StringTok}[1]{\textcolor[rgb]{0.31,0.60,0.02}{#1}}
\newcommand{\VariableTok}[1]{\textcolor[rgb]{0.00,0.00,0.00}{#1}}
\newcommand{\VerbatimStringTok}[1]{\textcolor[rgb]{0.31,0.60,0.02}{#1}}
\newcommand{\WarningTok}[1]{\textcolor[rgb]{0.56,0.35,0.01}{\textbf{\textit{#1}}}}
\usepackage{graphicx}
\makeatletter
\newsavebox\pandoc@box
\newcommand*\pandocbounded[1]{% scales image to fit in text height/width
  \sbox\pandoc@box{#1}%
  \Gscale@div\@tempa{\textheight}{\dimexpr\ht\pandoc@box+\dp\pandoc@box\relax}%
  \Gscale@div\@tempb{\linewidth}{\wd\pandoc@box}%
  \ifdim\@tempb\p@<\@tempa\p@\let\@tempa\@tempb\fi% select the smaller of both
  \ifdim\@tempa\p@<\p@\scalebox{\@tempa}{\usebox\pandoc@box}%
  \else\usebox{\pandoc@box}%
  \fi%
}
% Set default figure placement to htbp
\def\fps@figure{htbp}
\makeatother
\setlength{\emergencystretch}{3em} % prevent overfull lines
\providecommand{\tightlist}{%
  \setlength{\itemsep}{0pt}\setlength{\parskip}{0pt}}
\setcounter{secnumdepth}{-\maxdimen} % remove section numbering
\usepackage{bookmark}
\IfFileExists{xurl.sty}{\usepackage{xurl}}{} % add URL line breaks if available
\urlstyle{same}
\hypersetup{
  pdftitle={Mortality Investigation Report},
  pdfauthor={Moffat Kagiri},
  hidelinks,
  pdfcreator={LaTeX via pandoc}}

\title{Mortality Investigation Report}
\author{Moffat Kagiri}
\date{2025-08-13}

\begin{document}
\maketitle

\section{Introduction}\label{introduction}

This is an exercise to study mortality investigation. It is executed in
R, validated against the answers provided in the worked example, and
documented in the actuarial work repository on github. The goal is to:

\item

Calculate crude and graduated mortality rates

\item

Compare actual vs expected experience

\item

Provide diagnostic plots

\item

Document the process as a reliable reference point in future

\section{Data Preparation}\label{data-preparation}

The data is availed as a .csv file named FuneralData.csv Let's do an
exploratory analysis.

\begin{Shaded}
\begin{Highlighting}[]
\NormalTok{main }\OtherTok{\textless{}{-}} \FunctionTok{read.csv}\NormalTok{(}\StringTok{"C:/Users/moffat.kagiri/models/mortality investigation/FuneralData.csv"}\NormalTok{, }\AttributeTok{header =} \ConstantTok{TRUE}\NormalTok{)}
\FunctionTok{head}\NormalTok{(main)}
\end{Highlighting}
\end{Shaded}

\begin{verbatim}
##   LIFE    BIRTH    ENTRY    DEATH
## 1    1 1944.089 2014.183 2017.358
## 2    2 1946.778 2017.050 2017.683
## 3    3 1946.931 2012.344       NA
## 4    4 1953.486 2012.322 2016.317
## 5    5 1967.156 2017.192       NA
## 6    6 1957.681 2017.025       NA
\end{verbatim}

\begin{Shaded}
\begin{Highlighting}[]
\FunctionTok{summary}\NormalTok{(main)}
\end{Highlighting}
\end{Shaded}

\begin{verbatim}
##       LIFE            BIRTH          ENTRY          DEATH     
##  Min.   :   1.0   Min.   :1936   Min.   :2010   Min.   :2011  
##  1st Qu.: 250.8   1st Qu.:1950   1st Qu.:2012   1st Qu.:2015  
##  Median : 500.5   Median :1958   Median :2014   Median :2017  
##  Mean   : 500.5   Mean   :1956   Mean   :2014   Mean   :2016  
##  3rd Qu.: 750.2   3rd Qu.:1963   3rd Qu.:2016   3rd Qu.:2018  
##  Max.   :1000.0   Max.   :1968   Max.   :2018   Max.   :2018  
##                                                 NA's   :847
\end{verbatim}

\subsection{Optional Block}\label{optional-block}

Start date t\_0 = 2013.00 and end date t\_4 \(\approx\) 2018. Therefore,
we can define age relative to each year in the study, for each life.

\begin{Shaded}
\begin{Highlighting}[]
\CommentTok{\# Define the start and end dates as per the format in the data}
\NormalTok{t\_0 }\OtherTok{\textless{}{-}} \FunctionTok{round}\NormalTok{(}\FunctionTok{decimal\_date}\NormalTok{(}\FunctionTok{ymd}\NormalTok{(}\StringTok{"2013{-}01{-}01"}\NormalTok{)), }\DecValTok{4}\NormalTok{)}
\NormalTok{t\_5 }\OtherTok{\textless{}{-}} \FloatTok{2017.999} \CommentTok{\#round(decimal\_date(ymd("2017{-}12{-}31")), 4)}
\FunctionTok{print}\NormalTok{(}\FunctionTok{c}\NormalTok{(t\_0, t\_5))}
\end{Highlighting}
\end{Shaded}

\begin{verbatim}
## [1] 2013.000 2017.999
\end{verbatim}

The three dates whose maximum determines the date at which exposure at
age 70 begins are:

\item

The date of reaching age label 70

\item

The date of entry

\item

The start of the investigation

The three dates whose minimum determines the date at which exposure at
age 70 ends are:

\item

The date of reaching age label 71

\item

The date of exit (for any reason)

\item

The end of the investigation

To determine the exact central exposed to risk for age 70 last birthday,
the block below manually determines the exposure as the duration during
which lives under observation have age label 70. This is the
intersection of the interval between a life's 70th and 71st birthdays,
and the period of the investigation. This is calculated for every life
and then added up to show the total exposure as needed.

An important takeaway in this exercise was to avoid the use of ifelse
with date time formats. This loop strips the Date/POSIXct attributes as
well as the interval class attributes, making the entire exercise overly
frustrating.

\begin{Shaded}
\begin{Highlighting}[]
\CommentTok{\# Determine exact central exposed to risk for age 70 last birthday. }
\CommentTok{\# Define 70th and 71st birthdays}
\NormalTok{exact }\OtherTok{\textless{}{-}}\NormalTok{ main[}\DecValTok{1}\SpecialCharTok{:}\DecValTok{4}\NormalTok{]}
\NormalTok{exact}\SpecialCharTok{$}\NormalTok{birthday\_70 }\OtherTok{\textless{}{-}} \FunctionTok{date\_decimal}\NormalTok{(main}\SpecialCharTok{$}\NormalTok{BIRTH) }\SpecialCharTok{+} \FunctionTok{years}\NormalTok{(}\DecValTok{70}\NormalTok{)}
\NormalTok{exact}\SpecialCharTok{$}\NormalTok{birthday\_71 }\OtherTok{\textless{}{-}} \FunctionTok{date\_decimal}\NormalTok{(main}\SpecialCharTok{$}\NormalTok{BIRTH) }\SpecialCharTok{+} \FunctionTok{years}\NormalTok{(}\DecValTok{71}\NormalTok{)}

\CommentTok{\# Define other dates}
\NormalTok{exact}\SpecialCharTok{$}\NormalTok{entry\_date }\OtherTok{\textless{}{-}} \FunctionTok{date\_decimal}\NormalTok{(main}\SpecialCharTok{$}\NormalTok{ENTRY)}
\NormalTok{exact}\SpecialCharTok{$}\NormalTok{start\_date }\OtherTok{\textless{}{-}} \FunctionTok{date\_decimal}\NormalTok{(t\_0) }

\CommentTok{\# Replace NA in DEATH with t\_5, keeping Date format}
\NormalTok{exact}\SpecialCharTok{$}\NormalTok{exit\_date }\OtherTok{\textless{}{-}} \FunctionTok{as.Date}\NormalTok{(}\ConstantTok{NA}\NormalTok{)  }\CommentTok{\# Initialize as Date}
\NormalTok{exact}\SpecialCharTok{$}\NormalTok{exit\_date }\OtherTok{\textless{}{-}} \FunctionTok{if\_else}\NormalTok{(}
  \FunctionTok{is.na}\NormalTok{(main}\SpecialCharTok{$}\NormalTok{DEATH),}
  \FunctionTok{date\_decimal}\NormalTok{(t\_5),           }\CommentTok{\# Use study end date }
  \FunctionTok{date\_decimal}\NormalTok{(main}\SpecialCharTok{$}\NormalTok{DEATH)     }\CommentTok{\# Keep original death date}
\NormalTok{)}

\NormalTok{exact}\SpecialCharTok{$}\NormalTok{end\_date }\OtherTok{\textless{}{-}} \FunctionTok{date\_decimal}\NormalTok{(t\_5)}

\CommentTok{\# Calculate exposure using proper date{-}preserving methods}
\NormalTok{exact }\OtherTok{\textless{}{-}}\NormalTok{ exact }\SpecialCharTok{\%\textgreater{}\%}
  \FunctionTok{mutate}\NormalTok{(}
    \CommentTok{\# Step 1: Define exposure windows (already correct)}
    \AttributeTok{exposure\_start =} \FunctionTok{pmax}\NormalTok{(birthday\_70, entry\_date, start\_date),}
    \AttributeTok{exposure\_end =} \FunctionTok{pmin}\NormalTok{(birthday\_71, exit\_date, end\_date),}
    
    \CommentTok{\# Step 2: Compute exposure time safely}
    \AttributeTok{EXPOSURE\_AGE70 =} \FunctionTok{case\_when}\NormalTok{(}
\NormalTok{      exposure\_start }\SpecialCharTok{\textgreater{}=}\NormalTok{ exposure\_end }\SpecialCharTok{\textasciitilde{}} \DecValTok{0}\NormalTok{,  }\CommentTok{\# No overlap}
      \ConstantTok{TRUE} \SpecialCharTok{\textasciitilde{}} \FunctionTok{round}\NormalTok{(}\FunctionTok{time\_length}\NormalTok{(}\FunctionTok{interval}\NormalTok{(exposure\_start, exposure\_end), }\StringTok{"years"}\NormalTok{), }\DecValTok{4}\NormalTok{)}
\NormalTok{    )}
\NormalTok{  )}

\CommentTok{\# Sum exposure (now numeric, not interval)}
\NormalTok{total\_E70 }\OtherTok{\textless{}{-}} \FunctionTok{round}\NormalTok{(}\FunctionTok{sum}\NormalTok{(exact}\SpecialCharTok{$}\NormalTok{EXPOSURE\_AGE70, }\AttributeTok{na.rm =} \ConstantTok{TRUE}\NormalTok{), }\DecValTok{4}\NormalTok{)}
\FunctionTok{print}\NormalTok{(}\FunctionTok{paste}\NormalTok{(}\StringTok{"Total central exposed to risk at age 70:"}\NormalTok{, }\FunctionTok{round}\NormalTok{(total\_E70, }\DecValTok{4}\NormalTok{), }\StringTok{"person{-}years"}\NormalTok{))}
\end{Highlighting}
\end{Shaded}

\begin{verbatim}
## [1] "Total central exposed to risk at age 70: 70.4461 person-years"
\end{verbatim}

Without the date-time conversions, this is an attempt at a more
reproducible calculation, factoring out the age parameter as a variable.

Note to maintain the accuracy indicated in the question. The answer
swayed between 70.401 and 70.462 due to rounding off errors.

\begin{Shaded}
\begin{Highlighting}[]
\CommentTok{\# Determine exact central exposed to risk for age 70 last birthday. }
\CommentTok{\# Define 70th and 71st birthdays}

\NormalTok{x }\OtherTok{\textless{}{-}} \DecValTok{70}
\NormalTok{exact}\SpecialCharTok{$}\NormalTok{birthday\_x }\OtherTok{\textless{}{-}}\NormalTok{ main}\SpecialCharTok{$}\NormalTok{BIRTH }\SpecialCharTok{+}\NormalTok{ x}
\NormalTok{exact}\SpecialCharTok{$}\NormalTok{birthday\_x1 }\OtherTok{\textless{}{-}}\NormalTok{ main}\SpecialCharTok{$}\NormalTok{BIRTH }\SpecialCharTok{+}\NormalTok{ x }\SpecialCharTok{+} \DecValTok{1}

\CommentTok{\# Define other dates}
\NormalTok{exact}\SpecialCharTok{$}\NormalTok{entry\_date }\OtherTok{\textless{}{-}}\NormalTok{ main}\SpecialCharTok{$}\NormalTok{ENTRY}
\NormalTok{exact}\SpecialCharTok{$}\NormalTok{start\_date }\OtherTok{\textless{}{-}}\NormalTok{ t\_0 }

\CommentTok{\# Replace NA in DEATH with t\_5, keeping Date format}
\NormalTok{exact}\SpecialCharTok{$}\NormalTok{exit\_date }\OtherTok{\textless{}{-}} \ConstantTok{NA}  \CommentTok{\# Initialize as Date}
\NormalTok{exact}\SpecialCharTok{$}\NormalTok{exit\_date }\OtherTok{\textless{}{-}} \FunctionTok{if\_else}\NormalTok{(}
  \FunctionTok{is.na}\NormalTok{(main}\SpecialCharTok{$}\NormalTok{DEATH),}
\NormalTok{  t\_5,           }\CommentTok{\# Use study end date }
\NormalTok{  main}\SpecialCharTok{$}\NormalTok{DEATH     }\CommentTok{\# Keep original death date}
\NormalTok{)}

\NormalTok{exact}\SpecialCharTok{$}\NormalTok{end\_date }\OtherTok{\textless{}{-}}\NormalTok{ t\_5}

\CommentTok{\# Step 1: Define exposure windows (already correct)}
\NormalTok{exact}\SpecialCharTok{$}\NormalTok{exposure\_start }\OtherTok{\textless{}{-}} \FunctionTok{pmax}\NormalTok{(exact}\SpecialCharTok{$}\NormalTok{birthday\_x, exact}\SpecialCharTok{$}\NormalTok{entry\_date, exact}\SpecialCharTok{$}\NormalTok{start\_date)}
\NormalTok{exact}\SpecialCharTok{$}\NormalTok{exposure\_end }\OtherTok{\textless{}{-}} \FunctionTok{pmin}\NormalTok{(exact}\SpecialCharTok{$}\NormalTok{birthday\_x1, exact}\SpecialCharTok{$}\NormalTok{exit\_date, exact}\SpecialCharTok{$}\NormalTok{end\_date)}

\CommentTok{\# Step 2: Compute exposure time safely}
\NormalTok{exact}\SpecialCharTok{$}\NormalTok{EXPOSURE\_AGEX }\OtherTok{\textless{}{-}} \FunctionTok{case\_when}\NormalTok{(}
\NormalTok{  exact}\SpecialCharTok{$}\NormalTok{exposure\_start }\SpecialCharTok{\textgreater{}=}\NormalTok{ exact}\SpecialCharTok{$}\NormalTok{exposure\_end }\SpecialCharTok{\textasciitilde{}} \DecValTok{0}\NormalTok{,  }\CommentTok{\# No overlap}
  \ConstantTok{TRUE} \SpecialCharTok{\textasciitilde{}}  \FunctionTok{round}\NormalTok{(exact}\SpecialCharTok{$}\NormalTok{exposure\_end }\SpecialCharTok{{-}}\NormalTok{ exact}\SpecialCharTok{$}\NormalTok{exposure\_start, }\DecValTok{6}\NormalTok{))}

\CommentTok{\# Sum exposure (now numeric, not interval)}
\NormalTok{total\_EX }\OtherTok{\textless{}{-}} \FunctionTok{sum}\NormalTok{(exact}\SpecialCharTok{$}\NormalTok{EXPOSURE\_AGEX, }\AttributeTok{na.rm =} \ConstantTok{TRUE}\NormalTok{)}
\FunctionTok{print}\NormalTok{(}\FunctionTok{paste}\NormalTok{(}\StringTok{"Total central exposed to risk at age 70:"}\NormalTok{, }\FunctionTok{round}\NormalTok{(total\_EX, }\DecValTok{4}\NormalTok{), }\StringTok{"person{-}years"}\NormalTok{))}
\end{Highlighting}
\end{Shaded}

\begin{verbatim}
## [1] "Total central exposed to risk at age 70: 70.444 person-years"
\end{verbatim}

The solution provided in the worked example is:

\subsection{Calculate the number of lives who died at age 70 last
birthday during the investigation
period.}\label{calculate-the-number-of-lives-who-died-at-age-70-last-birthday-during-the-investigation-period.}

The data includes dates of death, enabling us to determine the lives who
died. From the dates of death, we can determine the age last birthday
for each life at the point of death during the study.

\begin{Shaded}
\begin{Highlighting}[]
\CommentTok{\# Number of lives who died during observation}
\NormalTok{deaths }\OtherTok{\textless{}{-}}\NormalTok{ main[}\SpecialCharTok{!}\FunctionTok{is.na}\NormalTok{(main}\SpecialCharTok{$}\NormalTok{DEATH) }\SpecialCharTok{\&}
\NormalTok{                 main}\SpecialCharTok{$}\NormalTok{DEATH }\SpecialCharTok{\textgreater{}=} \FloatTok{2013.000} \SpecialCharTok{\&}
\NormalTok{                 main}\SpecialCharTok{$}\NormalTok{DEATH }\SpecialCharTok{\textless{}=} \FloatTok{2017.999}\NormalTok{, }\DecValTok{1}\SpecialCharTok{:}\DecValTok{4}\NormalTok{]}
\FunctionTok{summary}\NormalTok{(deaths)}
\end{Highlighting}
\end{Shaded}

\begin{verbatim}
##       LIFE           BIRTH          ENTRY          DEATH     
##  Min.   :  1.0   Min.   :1937   Min.   :2010   Min.   :2013  
##  1st Qu.:227.0   1st Qu.:1945   1st Qu.:2012   1st Qu.:2015  
##  Median :476.0   Median :1953   Median :2014   Median :2017  
##  Mean   :482.4   Mean   :1953   Mean   :2014   Mean   :2016  
##  3rd Qu.:741.0   3rd Qu.:1962   3rd Qu.:2015   3rd Qu.:2017  
##  Max.   :997.0   Max.   :1966   Max.   :2018   Max.   :2018
\end{verbatim}

\begin{Shaded}
\begin{Highlighting}[]
\CommentTok{\# Determine age last birthday at death}
\NormalTok{deaths}\SpecialCharTok{$}\NormalTok{AGE }\OtherTok{\textless{}{-}} \FunctionTok{floor}\NormalTok{(deaths}\SpecialCharTok{$}\NormalTok{DEATH }\SpecialCharTok{{-}}\NormalTok{ deaths}\SpecialCharTok{$}\NormalTok{BIRTH)}

\CommentTok{\# Number of lives who died at age 70 last birthday is:}
\NormalTok{num }\OtherTok{=} \FunctionTok{sum}\NormalTok{(deaths}\SpecialCharTok{$}\NormalTok{AGE }\SpecialCharTok{==} \DecValTok{70}\NormalTok{)}

\FunctionTok{print}\NormalTok{(num)}
\end{Highlighting}
\end{Shaded}

\begin{verbatim}
## [1] 3
\end{verbatim}

\section{Force of mortality}\label{force-of-mortality}

Force of mortality at age 70 is given by:

\begin{Shaded}
\begin{Highlighting}[]
\NormalTok{mu\_x }\OtherTok{=}\NormalTok{ num}\SpecialCharTok{/}\NormalTok{total\_EX}
\FunctionTok{print}\NormalTok{(mu\_x)}
\end{Highlighting}
\end{Shaded}

\begin{verbatim}
## [1] 0.04258702
\end{verbatim}

\subsection{Number of lives aged 70 last
birthday}\label{number-of-lives-aged-70-last-birthday}

To determine the number of lives aged 70 whose policies were in force at
the start of the investigation period, we check age last birthday at the
start of the investigation period.

Filter for: 1. Either still alive (NA in DEATH) OR died after
investigation start (DEATH \textgreater= t\_0) 2. AND was age 70 last
birthday at ENTRY 3. AND ENTRY was before/at investigation start (t\_0)

\begin{Shaded}
\begin{Highlighting}[]
\CommentTok{\# Define age last{-}birthday at the start of the investigation}
\NormalTok{main}\SpecialCharTok{$}\NormalTok{AGE }\OtherTok{\textless{}{-}} \FunctionTok{floor}\NormalTok{(t\_0 }\SpecialCharTok{{-}}\NormalTok{ main}\SpecialCharTok{$}\NormalTok{BIRTH)}
\NormalTok{p\_70\_start }\OtherTok{\textless{}{-}}\NormalTok{ main[main}\SpecialCharTok{$}\NormalTok{AGE }\SpecialCharTok{==} \DecValTok{70} \SpecialCharTok{\&}
\NormalTok{                     main}\SpecialCharTok{$}\NormalTok{ENTRY }\SpecialCharTok{\textless{}}\NormalTok{ t\_0 }\SpecialCharTok{\&}
\NormalTok{                     (}\FunctionTok{is.na}\NormalTok{(main}\SpecialCharTok{$}\NormalTok{DEATH) }\SpecialCharTok{|}\NormalTok{ main}\SpecialCharTok{$}\NormalTok{DEATH }\SpecialCharTok{\textgreater{}}\NormalTok{ t\_0), }\DecValTok{1}\SpecialCharTok{:}\DecValTok{4}\NormalTok{]}
\end{Highlighting}
\end{Shaded}

Determine the corresponding figures for 1 January in years 2014, 2015,
2016, 2017 and 2018. For this, we replicate the code above to regenerate
the same figure for different years.

\begin{Shaded}
\begin{Highlighting}[]
\CommentTok{\# Replicating to get the same figure for years from 2013 to 2018}

\NormalTok{p\_70 }\OtherTok{\textless{}{-}} \ControlFlowTok{function}\NormalTok{(date\_0)\{}
\NormalTok{  agex }\OtherTok{\textless{}{-}} \FunctionTok{floor}\NormalTok{(date\_0 }\SpecialCharTok{{-}}\NormalTok{ main}\SpecialCharTok{$}\NormalTok{BIRTH)}
\NormalTok{  lives }\OtherTok{\textless{}{-}}\NormalTok{ main[}
\NormalTok{    (}\FunctionTok{is.na}\NormalTok{(main}\SpecialCharTok{$}\NormalTok{DEATH) }\SpecialCharTok{|}\NormalTok{ main}\SpecialCharTok{$}\NormalTok{DEATH }\SpecialCharTok{\textgreater{}}\NormalTok{ date\_0) }\SpecialCharTok{\&} 
\NormalTok{      agex }\SpecialCharTok{==} \DecValTok{70} \SpecialCharTok{\&}
\NormalTok{      main}\SpecialCharTok{$}\NormalTok{ENTRY }\SpecialCharTok{\textless{}=}\NormalTok{ date\_0,}
             \DecValTok{1}\SpecialCharTok{:}\DecValTok{4}
\NormalTok{  ]}
  \FunctionTok{return}\NormalTok{(}\FunctionTok{nrow}\NormalTok{(lives))}
\NormalTok{\}}


\CommentTok{\# Create a vector of start dates for each year}
\NormalTok{year\_starts }\OtherTok{\textless{}{-}} \FunctionTok{decimal\_date}\NormalTok{(}\FunctionTok{ymd}\NormalTok{(}\FunctionTok{paste0}\NormalTok{(}\DecValTok{2013}\SpecialCharTok{:}\DecValTok{2018}\NormalTok{, }\StringTok{"{-}01{-}01"}\NormalTok{)))}

\CommentTok{\# Calculate and store results}
\NormalTok{results }\OtherTok{\textless{}{-}} \FunctionTok{sapply}\NormalTok{(year\_starts, p\_70)}

\CommentTok{\# Print the results}
\FunctionTok{print}\NormalTok{(}\StringTok{"The number of lives aged 70 whose policies were in force at the beginning of each year are:"}\NormalTok{)}
\end{Highlighting}
\end{Shaded}

\begin{verbatim}
## [1] "The number of lives aged 70 whose policies were in force at the beginning of each year are:"
\end{verbatim}

\begin{Shaded}
\begin{Highlighting}[]
\ControlFlowTok{for}\NormalTok{ (i }\ControlFlowTok{in} \DecValTok{1}\SpecialCharTok{:}\FunctionTok{length}\NormalTok{(results)) \{}
  \FunctionTok{print}\NormalTok{(}\FunctionTok{paste}\NormalTok{(}\FunctionTok{year}\NormalTok{(}\FunctionTok{ymd}\NormalTok{(}\FunctionTok{paste0}\NormalTok{(}\DecValTok{2012}\SpecialCharTok{+}\NormalTok{i, }\StringTok{"{-}01{-}01"}\NormalTok{))), }\StringTok{":"}\NormalTok{, results[i]))}
\NormalTok{\}}
\end{Highlighting}
\end{Shaded}

\begin{verbatim}
## [1] "2013 : 12"
## [1] "2014 : 9"
## [1] "2015 : 18"
## [1] "2016 : 14"
## [1] "2017 : 14"
## [1] "2018 : 19"
\end{verbatim}

\section{Census Approach}\label{census-approach}

Estimate the central exposed to risk for age 70 last birthday for the
investigation period using a census approach. For a population observed
between time \(t_1\) and \(t_2\):

\[ E^c_x = \int_{t_1}^{t_2} P_x(t) \, dt \] where:

\(P_x(t)\) = Number of lives aged \(x\) at time \(t\)

Integrate over the observation period.

In our case, \[ E^c_{70} = \int_{t_0}^{t_5} P_{70}(t) \, dt \] Thus

\[ E^c_{70} =\sum_{t=0}^5 \frac 1 2 \left( P_{70}(t)+P_{70}(t+1)) \right \]

This becomes:
\[E^c_{70} = \frac 1 2(12+9) + \frac 1 2(9+18) + \frac 1 2(18+14) + \frac 1 2(14+14) +\frac 1 2(14+19)\]

Therefore:

\[ E^c_{70} = 10.5+13.5+16+14+16.5 \] And \[E^c_{70} = 70.5 \]

Trying the trapezoid rule in R\ldots{}

\begin{Shaded}
\begin{Highlighting}[]
\CommentTok{\# Calculate central exposed to risk (census trapezoidal rule)}
\NormalTok{ce\_x }\OtherTok{\textless{}{-}} \ControlFlowTok{function}\NormalTok{(dates, populations) \{}
\NormalTok{  time\_intervals }\OtherTok{\textless{}{-}} \FunctionTok{as.numeric}\NormalTok{(}\FunctionTok{diff}\NormalTok{(dates)}\SpecialCharTok{/}\FloatTok{365.25}\NormalTok{)}
\NormalTok{  avg\_populations }\OtherTok{\textless{}{-}}\NormalTok{ (populations[}\SpecialCharTok{{-}}\DecValTok{1}\NormalTok{] }\SpecialCharTok{+}\NormalTok{ populations[}\SpecialCharTok{{-}}\FunctionTok{length}\NormalTok{(populations)]) }\SpecialCharTok{/} \DecValTok{2}
  \FunctionTok{sum}\NormalTok{(time\_intervals }\SpecialCharTok{*}\NormalTok{ avg\_populations)}
\NormalTok{\}}

\CommentTok{\# Example usage}
\NormalTok{c\_dates }\OtherTok{\textless{}{-}} \FunctionTok{as.Date}\NormalTok{(}\FunctionTok{date\_decimal}\NormalTok{(year\_starts))}
\NormalTok{populations }\OtherTok{\textless{}{-}}\NormalTok{ results}
\end{Highlighting}
\end{Shaded}

\$E\^{}c\_\{70\} = \$ \texttt{ce\_x(c\_dates,\ populations)}.

\#Estimated force of mortality (census method) As before, force of
mortality is given by: \[\mu_x = \frac{d_x}{E^c_x}\]

Which translates to: \[\mu_x = \frac{3}{70.5}\]

Therefore: \[\mu_x = 0.04255319\]

\end{document}
